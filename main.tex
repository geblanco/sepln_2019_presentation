\documentclass[]{beamer}
\usepackage{pgfpages}

% \setbeameroption{show notes on second screen}

% Include any extra LaTeX packages required
\usepackage[
  backend=bibtex,
  style=alphabetic,
  citestyle=authoryear
]{biblatex}

\addbibresource{./library.bib}

%% Packages
  \usepackage{listings}
  \usepackage{multicol}
  \usepackage{booktabs}
  \usepackage{xcolor}
  \usepackage{times}
  \usepackage{tikz}
  \usepackage{amsmath}
  \usepackage{verbatim}
  \usetikzlibrary{arrows.meta,shapes}

\usetheme{metropolis}

%% Settings and new commands
  \def\mAlertSpace{\vspace{0.5em}}
  \def\mOrangeItem{\item[\textcolor{mLightBrown}{\textbullet}]}
  % 36616a
  \definecolor{feldgrau}{rgb}{0.3, 0.36, 0.33}
  \newcommand{\mSlideTitle}{{{\color{gray}\secname}} \# \subsecname \hfill {\scriptsize$|$ UNED}}
  \newcommand{\mShortTitle}{\secname \hfill {\scriptsize$|$ UNED}}

  \tikzstyle{every picture}+=[remember picture]
  \everymath{\displaystyle}
  \input{json_listing.tex}

%% Preamble
  \title{Benchmarking Entity Linking for Question Answering over Knowledge Graphs}
  \author{\textbf{Guillermo Echegoyen Blanco} \\ \'Alvaro Rodrigo \\ Anselmo Pe\~nas \\ 
    \{gblanco, alvarory, anselmo\} \textbf{at} lsi.uned.es \\ \\
    NLP \& IR Group \\
    Universidad Nacional de Educaci\'on a Distancia}
    \titlegraphic{\hfill\includegraphics[height=0.75cm]{./figures/UNED.jpeg}
  }
  \date{}

\begin{document}

\maketitle

\begin{frame}{Overview}
  \setbeamertemplate{section in toc}[sections numbered]
  \tableofcontents[hideallsubsections]
\end{frame}

\section{Introduction}
  \begin{frame}{\mShortTitle}
    \begin{alertblock}{Entity Linking}
      \mAlertSpace%
      \textbf{Def:} Link parts of a Natural Language passage to their corresponding node in a Knowledge Graph. Usually comprises:
      \begin{itemize}
        \item Recognize the entity mention in the text.
        \item Disambiguate the mention.
      \end{itemize}
    \end{alertblock}
    \mAlertSpace
    \textbf{Q:} \textcolor{feldgrau}{List all episodes of the}
  \tikz[baseline]{\node[anchor=base] (classt1){%
    $\overbrace{\text{first season}}^\text{\scriptsize dbo:seasonNumber = 1}$
  };}
\textcolor{feldgrau}{of the HBO \\ television }
  \tikz[baseline]{\node[anchor=base] (classt2){%
    $\underbrace{\text{series}}_\text{\scriptsize dbo:series}$ 
  };}
  \tikz[baseline]{\node[anchor=base] (entityt1){%
    $\underbrace{\text{The Sopranos}}_\text{\scriptsize dbr:The\_Sopranos}$
  };}
\\


    \mAlertSpace
        \begin{tabular}{lr}
      {\scriptsize
        \textit{dbo = http://dbpedia.org/ontology}
      } &
      \hspace{3cm} 
      {%
        \tikz[baseline]{\node[anchor=base] (entityf1) {entity};}
      } \\
      {\scriptsize
        \textit{dbr = http://dbpedia.org/resource}
      } & {%
        \tikz[baseline]{\node[anchor=base] (classf1) {class};}
      } \\
    \end{tabular}

    \begin{tikzpicture}[overlay]
      \path[-{Latex[width=2mm]}] (classf1.east) edge [bend right=90] (classt1);
      \path[-{Latex[width=2mm]}] (classf1) edge [bend left=15] (classt2);
      \path[-{Latex[width=2mm]}] (entityf1) edge [bend left] (entityt1);
    \end{tikzpicture}
    \setbeamertemplate{footline}[plain]
  \end{frame}
  \note{%
    \begin{itemize}
      \item We frame it under a QA Task over a KG, although can be done anywhere.
    \end{itemize}
  }

  \begin{frame}{\mShortTitle}
    \begin{alertblock}{Motivation}
      \begin{itemize}
        \item Lots of QA systems do perform an EL step with good results.
        \item Asses impact of EL Task on QA systems over KG.
      \end{itemize}
    \end{alertblock}
  \end{frame}
  \note{%
    \begin{itemize}
      \item Lots of systems rely on other tools ToDo := Get some of those tools
      \item It is important because linking to a certain node will restrict the whole search process inside the KG. So it's impact can be decisive.
      \item More about the easyness of each dataset on the experiments section.
    \end{itemize}
  }

\section{Characterization}
  \begin{frame}{\mShortTitle}
    \begin{alertblock}{Input Datasets}
      \begin{itemize}
        \item QALD \{1-4\} \cite{Unger2014}) $\le$ 200 QA pairs each
        \item LC-QuAD (\cite{trivedi2017lc}) 5K QA pairs
      \end{itemize}
    \end{alertblock}
    % \begin{alertblock}{Example}
    %   \mAlertSpace
    %   \textbf{Q:} List all episodes of the first season of the HBO television series
    %   $\underbrace{\text{The Sopranos}}_\text{\scriptsize dbr:The\_Sopranos}$
    % \end{alertblock}
  \end{frame}
  \note{%
    \begin{itemize}
      \item There are other mentions to link (series, sesason Number) but correspond to classes.
    \end{itemize}
  }

  \begin{frame}[fragile]{\mShortTitle}
    \begin{alertblock}{Example}
      \begin{lstlisting}[
    language=json,
    basicstyle=\normalfont\ttfamily\scriptsize,
    label=list:QALD_dataset_sample,
    breakatwhitespace=false,
  ]
{
  "id": "37",
  "query": { "sparql": "SELECT ?uri ... },
  "answers": {
    "answer": [{ ...
    }, ...]
  },
  "question": [
    {
      "string": "List all episodes of the first season of the HBO television series The Sopranos!",
      "language": "en"
    }
  ]
}
\end{lstlisting}

    \end{alertblock}
  \end{frame}

  \begin{frame}{\mShortTitle}
    \begin{alertblock}{Difficulty?}
      \begin{itemize}
        \item Given the Question, how easy is the Entity Linking?
      \end{itemize}
      {\centering
\begin{tabular}{l|c|c|c|c}
\textbf{EL Casuistry}  & \textbf{QALD-1} & \textbf{QALD-2} & \textbf{QALD-3} & \textbf{QALD-4} \\
\hspace{2cm}Total      & \textbf{15}     & \textbf{84}     & \textbf{89}     & \textbf{202} \\ \hline \hline
Identical to DBP uri   & 73.33           & 85.71           & 84.27           & 79.21           \\ \hline
Missing tokens         &                 & 4.76            & 5.62            & 9.9             \\ \hline
Additional tokens      & 20.0            & 1.19            & 1.12            & 0.5             \\ \hline
Lexical variation      & 6.67            & 5.95            & 5.62            & 8.42            \\ \hline
Other                  &                 & 2.38            & 3.37            & 1.98            \\ \hline
\end{tabular}
}

    \end{alertblock}
  \end{frame}
  \note{%
    \begin{itemize}
      \item Numbers are from Unique Entities
    \end{itemize}
    QALD-4
    Tools: gAnswer (Zou), CASIA (Shizhu), POMELO (Hamon)
    Creation: Taken from QALD-3 and added more questions manually, QALD-3 was manually crafted.
    Use until QALD-4 beacuse 5 and 6th versions are more into multilang and were collected by prunning QALD-{1-4}. QALD-7 features only 214 questions.
    LC-QuAD
    Creation: Entity seed list -> subgraph generation -> sql and question template fill ->  manual correction and review
  }
  \begin{frame}{\mShortTitle}
    \begin{alertblock}{Problem}
      \begin{itemize}
          \item Actual collections for QA are easy for Entity Linking.
      \end{itemize}
      \textbf{Q:} \textcolor{feldgrau}{List all episodes of the of the HBO television series }
  \tikz[baseline]{\node[anchor=base] (dst1){%
    $\underbrace{\text{The Sopranos}}_\text{\scriptsize dbr:The\_Sopranos}$
  };}
\\


      \hfill
      \tikz[baseline]{\node[anchor=base] (src1){%
          \textbf{replace " " $\rightarrow$ "\_"}
      };}
      \begin{tikzpicture}[overlay]
        \path[-{Latex[width=2mm]}] (src1.west) edge [bend left] (dst1.south);
      \end{tikzpicture}
    \end{alertblock}
  \end{frame}

\section{Automatic Generation}
  \begin{frame}{\mShortTitle}
    \textcolor{mLightBrown}{\textbf{Objective:}} Complex dataset for Entity Linking
    \begin{alertblock}{Strategy}
      \begin{enumerate}
        \item Develop method to detect easy mentions
        \item Remove easy mentions from collection
      \end{enumerate}
    \end{alertblock}
    \begin{alertblock}{Methods}
      \begin{itemize}
        \item Trigram based mention detection
        \item Distance based mention detection
      \end{itemize}
    \end{alertblock}
  \end{frame}
  \note{%
    Distance metric: Levenstein
    \begin{itemize}
      \item Trigram: Trigram window slide minimizing distance.
      \item Distance: Maximum span of text minimizing distance, same number of tokens as the database entity.
    \end{itemize}
  }

\section{Results}
  \begin{frame}{\mShortTitle}
    \mAlertSpace
    \begin{alertblock}{Released Datasets}
      % \begin{itemize}
      %   \item \textbf{QALD-\{1-4\}-EL}: QALD-X version for EL $\le 45$ samples each.
      %   \item \textbf{LC-QuAD-EL}: LC-QuAD version for EL $\le 1.3K$ samples.
      %   \item \textbf{C-EL4QA}: Complex compilation of EL versions $\le 1.5K$ samples.
      % \end{itemize}
      \begin{table}[h!]
  \centering
  \begin{tabular}{|c|r|r|r|}
    \hline\rule{-2pt}{15pt}
    \textbf{Dataset} & \textbf{Unique Q.} & \textbf{Unique E.} & \textbf{Total} \\
    \hline\rule{-4pt}{10pt}
    QALD-1-EL & 3 & 3 & 4 \\ \hline
    QALD-2-EL & 11 & 11 & 12 \\ \hline
    QALD-3-EL & 13 & 13 & 14 \\ \hline
    QALD-4-EL & 38 & 40 & 45 \\ \hline
    LC-QuAD-EL & 1204 & 997 & 1292 \\ \hline
    C-EL4QA & 1269 & 1064 & 1367 \\ \hline
  \end{tabular}
\end{table}

    \end{alertblock}
  \end{frame}
  \note{%
    \begin{itemize}
      \item QALD datasets shrink by a 50\%.
      \item LCQUAD shrinks by a 70\%
    \end{itemize}
    Obviously this happens because the questions are automatically generated.
  }

\section{Conclusions \& Future Work}
  \begin{frame}{\mShortTitle}
    \begin{alertblock}{Conclusions}
      \begin{itemize}
        \item We found QA that collections do not really tackle the EL problem.
        \item QA Systems go for automated solutions
      \end{itemize}
    \end{alertblock}
    \begin{alertblock}{Open Questions}
      \begin{itemize}
        \item If Entity Linking were more difficult, how QA system would perform?
      \end{itemize}
    \end{alertblock}
  \end{frame}
  \note{%
    \begin{itemize}
      \item Systems like the ones on LC-QuAD paper (Aqua QA ...) ToDo := Extract refs
    \end{itemize}
  }

\section{Outcomes}
  \begin{frame}{\mShortTitle}
    \begin{alertblock}{Our main contributions are:}
      \begin{itemize}
        \item QA Datasets characterization
        \item Semi-automatic method to generate complex EL datasets.
        \item Release large benchmark dataset and baseline for EL in QA (\href{https://github.com/m0n0l0c0/benchmarking_entity_linking/blob/master/datasets/Complex-EL4QA.json}{url})
      \end{itemize}
    \end{alertblock}      
  \end{frame}
  \note{%
    \begin{itemize}
      \item Method serves also a baseline to test other systems
    \end{itemize}
  }

\begin{frame}[standout]
  Thank you!\\
  Questions?
\end{frame}
\note{%
  Possible questions:
  \begin{itemize}
    \item Have you tried this with Class linking instead of Entity?
  \end{itemize}
}

\section{References}
  \begin{frame}[allowframebreaks]{References}
    \printbibliography%
  \end{frame}

\end{document}
