\documentclass[]{beamer}
\usepackage{pgfpages}

% \setbeameroption{show notes on second screen}

% Include any extra LaTeX packages required
\usepackage[
  backend=bibtex,
  style=alphabetic,
  citestyle=authoryear
]{biblatex}

\addbibresource{./library.bib}

\usepackage{multicol}
\usepackage{booktabs}
\usepackage{xcolor}
\usepackage{times}
\usepackage{tikz}
\usepackage{amsmath}
\usepackage{verbatim}
\usetikzlibrary{arrows.meta,shapes}

\usetheme{metropolis}

\title{Benchmarking Entity Linking for Question Answering over Knowledge Graphs}
\author{\textbf{Guillermo Echegoyen Blanco} \\ \'Alvaro Rodrigo \\ Anselmo Pe\~nas \\ 
\{gblanco, alvarory, anselmo\} \textbf{at} lsi.uned.es \\ \\
Universidad Nacional de Educaci\'on a Distancia}
\institute{
  \vspace{1cm}
  \includegraphics[width=1.25cm]{./figures/UNED.jpeg}
}
\date{}
% \date{\the\year}

\def\mAlertSpace{\vspace{0.5em}}
\def\mOrangeItem{\item[\textcolor{mLightBrown}{\textbullet}]}
\newcommand{\mSlideTitle}{{{\color{gray}\secname}} \# \subsecname}

\begin{document}

\tikzstyle{every picture}+=[remember picture]
\everymath{\displaystyle}

\maketitle

\begin{frame}{Overview}
  \setbeamertemplate{section in toc}[sections numbered]
  \tableofcontents[hideallsubsections]
\end{frame}

\section{Introduction}
  \begin{frame}{\secname}
    \begin{alertblock}{Entity Linking}
      \mAlertSpace%
      \textbf{Def:} Link parts of a Natural Language passage to it's corresponding node in a Knowledge Graph. Usually comprises:
      \begin{itemize}
        \item Recognize the entity mention in the text.
        \item Disambiguate the mention.
      \end{itemize}
    \end{alertblock}
    \mAlertSpace
    \textbf{Q:} List all episodes of the
      \tikz[baseline]{\node[anchor=base] (classt1){%
        $\overbrace{\text{first season}}^\text{\scriptsize dbo:seasonNumber = 1}$
      };}
    of the HBO \\
    television 
      \tikz[baseline]{\node[anchor=base] (classt2){%
        $\underbrace{\text{series}}_\text{\scriptsize dbo:series}$ 
      };}
      \tikz[baseline]{\node[anchor=base] (entityt1){%
        $\underbrace{\text{The Sopranos}}_\text{\scriptsize dbr:The\_Sopranos}$
      };}
    \\
    \mAlertSpace
    \begin{tabular}{lr}
      {\scriptsize
        \textit{dbo = http://dbpedia.org/ontology}
      } &
      \hspace{4cm} 
      {%
        \tikz[baseline]{\node[anchor=base] (entityf1) {entity};}
      } \\
      {\scriptsize
        \textit{dbr = http://dbpedia.org/resource}
      } & {%
        \tikz[baseline]{\node[anchor=base] (classf1) {class};}
      } \\
    \end{tabular}
    \begin{tikzpicture}[overlay]
      \path[-{Latex[width=2mm]}] (classf1.east) edge [bend right=90] (classt1);
      \path[-{Latex[width=2mm]}] (classf1) edge [bend left=15] (classt2);
      \path[-{Latex[width=2mm]}] (entityf1) edge [bend left] (entityt1);
    \end{tikzpicture}
  \end{frame}
  \note{%
    \begin{itemize}
      \item We frame it under a QA Task over a KG, although can be done anywhere.
    \end{itemize}
  }

  \begin{frame}{\secname}
    \begin{alertblock}{Motivation}
      \begin{itemize}
        \item Asses the impact of Entity Linking on a Question Answering task over a KG.
        \item Actual collections for QA are easy for Entity Linking.
      \end{itemize}
    \end{alertblock}
  \end{frame}
  \note{%
    \begin{itemize}
      \item It is important because linking to a certain node will restrict the whole search process inside the KG. So it's impact can be decisive.
      \item More about the easyness of each dataset on the experiments section.
    \end{itemize}
  }

\section{Outcomes}
  \begin{frame}{\secname}
    \begin{alertblock}{Our main contributions are:}
      \begin{itemize}
        \item QA Datasets characterization
        \item Semi-automatic method to generate EL dataset.
        \item Release large benchmark dataset and baseline for EL in QA.
      \end{itemize}
    \end{alertblock}      
  \end{frame}

\section{Experiments}
  \begin{frame}{\secname}
    \begin{alertblock}{Datasets}
      \begin{itemize}
        \item QALD \{1-4\} \cite{Unger2014}) $\le$ 200 QA pairs each
        \item LC-QuAD (\cite{trivedi2017lc}) 5K QA pairs
      \end{itemize}
    \end{alertblock}
    \begin{alertblock}{Example}
      \mAlertSpace
      \textbf{Q:} List all episodes of the first season of the HBO television series
      $\underbrace{\text{The Sopranos}}_\text{\scriptsize dbr:The\_Sopranos}$
    \end{alertblock}
  \end{frame}
  \note{%
    \begin{itemize}
      \item There are other mentions to link (series, sesason Number) but correspond to classes.
    \end{itemize}
  }

\subsection{Characterization}
  \begin{frame}{\mSlideTitle}
    \begin{alertblock}{Difficulty}
      \begin{itemize}
        \item Given the Question, how easy is the Entity Linking?
      \end{itemize}
      {\centering
\begin{tabular}{l|c|c|c|c}
\textbf{EL Casuistry}  & \textbf{QALD-1} & \textbf{QALD-2} & \textbf{QALD-3} & \textbf{QALD-4} \\
\hspace{2cm}Total      & \textbf{15}     & \textbf{84}     & \textbf{89}     & \textbf{202} \\ \hline \hline
Identical to DBP uri   & 73.33           & 85.71           & 84.27           & 79.21           \\ \hline
Missing tokens         &                 & 4.76            & 5.62            & 9.9             \\ \hline
Additional tokens      & 20.0            & 1.19            & 1.12            & 0.5             \\ \hline
Lexical variation      & 6.67            & 5.95            & 5.62            & 8.42            \\ \hline
Other                  &                 & 2.38            & 3.37            & 1.98            \\ \hline
\end{tabular}
}

    \end{alertblock}
  \end{frame}
  \note{%
    \begin{itemize}
      \item Numbers are from Unique Entities.
    \end{itemize}
  }

\subsection{Automatic Generation}
  \begin{frame}{\mSlideTitle}
    \textcolor{mLightBrown}{\textbf{Objective:}} Complex dataset for Entity Linking
    \begin{alertblock}{Strategy}
      \begin{enumerate}
        \item Develop baseline to detect as much mentions as possible
        \item Remove items from collection
      \end{enumerate}
    \end{alertblock}
    \begin{alertblock}{Baselines}
      \begin{itemize}
        \item Trigram based mention detection
        \item Distance based mention detection
      \end{itemize}
    \end{alertblock}
  \end{frame}
  \note{%
    Distance metric: Levenstein
    \begin{itemize}
      \item Trigram: Trigram window slide minimizing distance
      \item Distance: Maximum span of text minimizing distance
    \end{itemize}
  }

\section{Results}
\subsection{Dataset Release}
  \begin{frame}{\mSlideTitle}
    \begin{alertblock}{Released Datasets}
      \begin{itemize}
        \item \textbf{QALD-\{1-4\}-EL}: QALD-X version for EL $\le 45$ samples each.
        \item \textbf{LC-QuAD-EL}: LC-QuAD version for EL $\le 1.3K$ samples.
        \item \textbf{C-EL4QA}: Compilation of EL versions $\le 1.5K$ samples.
      \end{itemize}
      \mAlertSpace
      Baseline removed 70\% of the dataset in the worst case!
    \end{alertblock}
  \end{frame}
  \note{%
    \begin{itemize}
      \item QALD datasets shrink by a 50\%.
      \item LCQUAD shrinks by a 70\%
    \end{itemize}
    Obviously this happens because the questions are automatically generated.
  }

\section{Discussion}
  \begin{frame}{\secname}
    \begin{alertblock}{Research Questions}
      \begin{itemize}
        \item If Entity Linking were more difficult, how QA system would perform?
        \item Datasets should be created more carefully.
      \end{itemize}
    \end{alertblock}
  \end{frame}
  \note{%
    \begin{itemize}
      \item Systems like the ones on LC-QuAD paper (Aqua QA ...) ToDo := Extract refs
    \end{itemize}
  }

  \begin{frame}[standout]
    Questions?
  \end{frame}
  \note{%
    Possible questions:
    \begin{itemize}
      \item Have you tried this with Class linking instead of Entity?
    \end{itemize}
  }

\section{References}
\begin{frame}[allowframebreaks]{References}
  \printbibliography%
\end{frame}

\end{document}
